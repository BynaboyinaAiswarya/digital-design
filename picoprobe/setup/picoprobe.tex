This section enumerates the steps needed to create a debugger probe from a
Raspberry Pi Pico, called the Picoprobe. For further details refer to the
\href{https://www.raspberrypi.com/documentation/microcontrollers/raspberry-pi-pico.html#debugging-using-another-raspberry-pi-pico}{documentation}.

Note that all instructions in this chapter are for Linux systems running
Debian-based distributions only. For other operating systems refer to the above
link.

\begin{enumerate}
    \item Setup the Raspberry Pi Pico environment on your laptop by entering the
    following commands at a terminal window.

    \begin{lstlisting}
sudo apt update && sudo apt upgrade
sudo apt install pkg-config
cd
wget https://raw.githubusercontent.com/raspberrypi/pico-setup/master/pico_setup.sh
chmod +x ./pico_setup.sh
SKIP_VSCODE=1 INCLUDE_PICOPROBE=1 ./pico_setup.sh
    \end{lstlisting}

    \item Download the picoprobe UF2 file from this 
    \href{https://www.raspberrypi.com/documentation/microcontrollers/raspberry-pi-pico.html#debugging-using-another-raspberry-pi-pico}{link}.
    \item Connect the Raspberry Pi Pico board to be used as the debugger
    (henceforth referred to as the ``debugger'') to your laptop and
    simultaneously press the BOOTSEL button to boot the debugger in bootloader
    mode.
    \item Flash the downloaded \texttt{picoprobe.uf2} file using
    \texttt{picotool} to the debugger as follows.
    \begin{lstlisting}
picotool save /path/to/picoprobe.uf2
    \end{lstlisting}
\end{enumerate}

This will convert the Raspberry Pi Pico board into a Picoprobe debugger which we
can use to flash and debug programs on other boards.